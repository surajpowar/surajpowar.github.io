% --- LaTeX Lecture Notes Template - S. Venkatraman ---

% --- Set document class and font size ---

\documentclass[letterpaper, 12pt]{article}

% --- Package imports ---

% Extended set of colors
\usepackage[dvipsnames]{xcolor}

\usepackage{
  amsmath, amsthm, amssymb, mathtools, dsfont, units,          % Math typesetting
  graphicx, wrapfig, subfig, float,                            % Figures and graphics formatting
  listings, color, inconsolata, pythonhighlight,               % Code formatting
  fancyhdr, sectsty, hyperref, enumerate, enumitem, framed }   % Headers/footers, section fonts, links, lists
\usepackage{mathrsfs}
\usepackage{calrsfs}
% lipsum is just for generating placeholder text and can be removed
\usepackage{hyperref, lipsum} 

% --- Fonts ---

%\usepackage{newpxtext, newpxmath, inconsolata}

% --- Page layout settings ---

% Set page margins
\usepackage[left=1.35in, right=1.35in, top=1.0in, bottom=.9in, headsep=.2in, footskip=0.35in]{geometry}

% Anchor footnotes to the bottom of the page
\usepackage[bottom]{footmisc}

% Set line spacing
\renewcommand{\baselinestretch}{1.2}

% Set spacing between paragraphs
\setlength{\parskip}{1.3mm}

% Allow multi-line equations to break onto the next page
\allowdisplaybreaks

% --- Page formatting settings ---

% Set image captions to be italicized
\usepackage[font={it,footnotesize}]{caption}

% Set link colors for labeled items (blue), citations (red), URLs (orange)
\hypersetup{colorlinks=true, linkcolor=RoyalBlue, citecolor=RedOrange, urlcolor=ForestGreen}

% Set font size for section titles (\large) and subtitles (\normalsize) 
\usepackage{titlesec}
\titleformat{\section}{\large\bfseries}{{\fontsize{19}{19}\selectfont\textreferencemark}\;\; }{0em}{}
\titleformat{\subsection}{\normalsize\bfseries\selectfont}{\thesubsection\;\;\;}{0em}{}

% Enumerated/bulleted lists: make numbers/bullets flush left
%\setlist[enumerate]{wide=2pt, leftmargin=16pt, labelwidth=0pt}
\setlist[itemize]{wide=0pt, leftmargin=16pt, labelwidth=10pt, align=left}

% --- Table of contents settings ---

\usepackage[subfigure]{tocloft}

% Reduce spacing between sections in table of contents
\setlength{\cftbeforesecskip}{.9ex}

% Remove indentation for sections
\cftsetindents{section}{0em}{0em}

% Set font size (\large) for table of contents title
\renewcommand{\cfttoctitlefont}{\large\bfseries}

% Remove numbers/bullets from section titles in table of contents
\makeatletter
\renewcommand{\cftsecpresnum}{\begin{lrbox}{\@tempboxa}}
\renewcommand{\cftsecaftersnum}{\end{lrbox}}
\makeatother

% --- Set path for images ---

\graphicspath{{Images/}{../Images/}}

% --- Math/Statistics commands ---

% Add a reference number to a single line of a multi-line equation
% Usage: "\numberthis\label{labelNameHere}" in an align or gather environment
\newcommand\numberthis{\addtocounter{equation}{1}\tag{\theequation}}

% Shortcut for bold text in math mode, e.g. $\b{X}$
\let\b\mathbf

% Shortcut for bold Greek letters, e.g. $\bg{\beta}$
\let\bg\boldsymbol

% Shortcut for calligraphic script, e.g. %\mc{M}$
\let\mc\mathcal

% \mathscr{(letter here)} is sometimes used to denote vector spaces
\usepackage[mathscr]{euscript}

% Convergence: right arrow with optional text on top
% E.g. $\converge[p]$ for converges in probability
\newcommand{\converge}[1][]{\xrightarrow{#1}}

% Weak convergence: harpoon symbol with optional text on top
% E.g. $\wconverge[n\to\infty]$
\newcommand{\wconverge}[1][]{\stackrel{#1}{\rightharpoonup}}

% Equality: equals sign with optional text on top
% E.g. $X \equals[d] Y$ for equality in distribution
\newcommand{\equals}[1][]{\stackrel{\smash{#1}}{=}}

% Normal distribution: arguments are the mean and variance
% E.g. $\normal{\mu}{\sigma}$
\newcommand{\normal}[2]{\mathcal{N}\left(#1,#2\right)}

% Uniform distribution: arguments are the left and right endpoints
% E.g. $\unif{0}{1}$
\newcommand{\unif}[2]{\text{Uniform}(#1,#2)}

% Independent and identically distributed random variables
% E.g. $ X_1,...,X_n \iid \normal{0}{1}$
\newcommand{\iid}{\stackrel{\smash{\text{iid}}}{\sim}}

% Sequences (this shortcut is mostly to reduce finger strain for small hands)
% E.g. to write $\{A_n\}_{n\geq 1}$, do $\bk{A_n}{n\geq 1}$
\newcommand{\bk}[2]{\{#1\}_{#2}}

% Math mode symbols for common sets and spaces. Example usage: $\R$
\newcommand{\R}{\mathbb{R}}	% Real numbers
\newcommand{\C}{\mathbb{C}}	% Complex numbers
\newcommand{\Q}{\mathbb{Q}}	% Rational numbers
\newcommand{\Z}{\mathbb{Z}}	% Integers
\newcommand{\N}{\mathbb{N}}	% Natural numbers
\newcommand{\F}{\mathcal{F}}	% Calligraphic F for a sigma algebra
\newcommand{\El}{\mathcal{L}}	% Calligraphic L, e.g. for L^p spaces

% Math mode symbols for probability
\newcommand{\pr}{\mathbb{P}}	% Probability measure
\newcommand{\E}{\mathbb{E}}	% Expectation, e.g. $\E(X)$
\newcommand{\var}{\text{Var}}	% Variance, e.g. $\var(X)$
\newcommand{\cov}{\text{Cov}}	% Covariance, e.g. $\cov(X,Y)$
\newcommand{\corr}{\text{Corr}}	% Correlation, e.g. $\corr(X,Y)$
\newcommand{\B}{\mathcal{B}}	% Borel sigma-algebra

% Other miscellaneous symbols
\newcommand{\tth}{\text{th}}	% Non-italicized 'th', e.g. $n^\tth$
\newcommand{\Oh}{\mathcal{O}}	% Big-O notation, e.g. $\O(n)$
\newcommand{\1}{\mathds{1}}	% Indicator function, e.g. $\1_A$

% Additional commands for math mode
\DeclareMathOperator*{\argmax}{argmax}		% Argmax, e.g. $\argmax_{x\in[0,1]} f(x)$
\DeclareMathOperator*{\argmin}{argmin}		% Argmin, e.g. $\argmin_{x\in[0,1]} f(x)$
\DeclareMathOperator*{\spann}{Span}		% Span, e.g. $\spann\{X_1,...,X_n\}$
\DeclareMathOperator*{\bias}{Bias}		% Bias, e.g. $\bias(\hat\theta)$
\DeclareMathOperator*{\ran}{ran}			% Range of an operator, e.g. $\ran(T) 
\DeclareMathOperator*{\dv}{d\!}			% Non-italicized 'with respect to', e.g. $\int f(x) \dv x$
\DeclareMathOperator*{\diag}{diag}		% Diagonal of a matrix, e.g. $\diag(M)$
\DeclareMathOperator*{\trace}{trace}		% Trace of a matrix, e.g. $\trace(M)$
\DeclareMathOperator*{\supp}{supp}		% Support of a function, e.g., $\supp(f)$

% Numbered theorem, lemma, etc. settings - e.g., a definition, lemma, and theorem appearing in that 
% order in Lecture 2 will be numbered Definition 2.1, Lemma 2.2, Theorem 2.3. 
% Example usage: \begin{theorem}[Name of theorem] Theorem statement \end{theorem}
\theoremstyle{definition}
\newtheorem{theorem}{Theorem}[section]
\newtheorem{proposition}[theorem]{Proposition}
\newtheorem{lemma}[theorem]{Lemma}
\newtheorem{corollary}[theorem]{Corollary}
\newtheorem{definition}[theorem]{Definition}
\newtheorem{example}[theorem]{Example}
\newtheorem{remark}[theorem]{Remark}

% Un-numbered theorem, lemma, etc. settings
% Example usage: \begin{lemma*}[Name of lemma] Lemma statement \end{lemma*}
\newtheorem*{theorem*}{Theorem}
\newtheorem*{proposition*}{Proposition}
\newtheorem*{lemma*}{Lemma}
\newtheorem*{corollary*}{Corollary}
\newtheorem*{definition*}{Definition}
\newtheorem*{example*}{Example}
\newtheorem*{remark*}{Remark}
\newtheorem*{claim}{Claim}

% --- Left/right header text (to appear on every page) ---

% Do not include a line under header or above footer
\pagestyle{fancy}
\renewcommand{\footrulewidth}{0pt}
\renewcommand{\headrulewidth}{0pt}

% Right header text: Lecture number and title
\renewcommand{\sectionmark}[1]{\markright{#1} }
\fancyhead[R]{\small\textit{\nouppercase{\rightmark}}}

% Left header text: Short course title, hyperlinked to table of contents
\fancyhead[L]{\hyperref[sec:contents]{\small Contents}} %Short title of document}}

% --- Document starts here ---

\begin{document}

% --- Main title and subtitle ---

\title{Math 501: Measure Theory Notes \\[1em]
\normalsize Based on class notes of Dr. Henri Schurz, SIU, Carbondale, IL}

% --- Author and date of last update ---

\author{\normalsize Suraj Powar}
\date{\normalsize\vspace{-1ex} Last updated: \today}

% --- Add title and table of contents ---

\maketitle
\tableofcontents\label{sec:contents}
\newpage
% --- Main content: import lectures as subfiles ---
\section[Chapter 1: Basics of Set Theory]{Basics of Set Theory}

\subsection{Elementary Sets Operations}
\definition{A \textbf{Set} is a well defined collection of items, things, objects, etc, and it is denoted by, }
\[S = \{s_i\}, \]
\text{ where the $s_i$ are called the elements of S}

\example{Following are the examples of sets.}
\begin{enumerate}
    \item $\mathbb{N}$ := \{0, 1, 2, ...\} \hspace*{\fill} (Natural Numbers) 
    \item $\mathbb{N_+}$ := \{1, 2, 3, ...\} \hspace*{\fill} (Positive Natural Numbers)
    \item $\mathbb{Z} := \{0, \pm{1}, \pm{2}, ...\}$ \hspace*{\fill} (Integers)
    \item $\mathbb{Q} := \left\{\dfrac{a}{b} \middle| \text{ a, b in } \mathbb{Z}; b \neq 0\right\}$ \hspace*{\fill} (Rational Numbers)
    \item $\mathbb{R} -$ Completion of $\mathbb{Q}$ with respect to metric $|x-y|$ \hspace*{\fill} (Real Numbers)
\end{enumerate}

Similarly for $\mathbb{N}^d, \mathbb{Z}^d, \mathbb{Q}^d, \mathbb{R}^d$ as d-dimensional extensions.

\definition{Let $f: dom(f) = X \to range(f) \subseteq Y$ be a function. The for all $A \subseteq dom(f)$, we denote \[f(A) := \{f(x) | x \in A\},\]
which is called as \textbf{image} of A under f. 
For all $B \subseteq range(f)$, we denote \[f^{-1}(B) := \{x \in dom(f) | f(x) \in B\}\]
which is called as \textbf{inverse image} of B under f (or simply \textbf{pre-image} of B under f). \\
The function f is said to be \textbf{onto} or \textbf{surjective} if and only if f(X) = Y. \\
The function f is said to be \textbf{one-to-one} or \textbf{injective} if and only if for all $x_1, x_2 \in dom(f)$, we have $f(x_1) = f(x_2) \to x_1 = x_2$.

\example{Any sequence $(x_n)_{n \in \mathbb{N}}$ where $x_n \in X$ may be viewed as a function }
\[f:= \mathbb{N} \to \text{ X via n}\in \mathbb{N} \to f(n) = x_n.\]

\definition{A set S is said to be \textbf{Countable} if and only if there exists a surjective mapping $f:= \mathbb{N} \to S$. Otherwise, S is said to be \textbf{Uncountable}.} \\
If the cardinality given as \#(S) of such a countable set S is infinite, we say that S is \textbf{Countably Infinite}. \\
If the cardinality \#(S) $< +\infty$ of the set S is infinite, then S is \textbf{Countably finite}. \\
We denote it by, \[\mathcal{P}(s) := \{X | X \subseteq S\} \]
\[\mathcal{P}(s) := (2^S) \]
and it is called as a \textbf{Power-set} of S. \\

\proposition{A Set S is countably infinite if and only if there exists a bijection $f: \mathbb{N} \to S.$}
\begin{proof}
    $\Longrightarrow$  \\
    By Definition, we know that an Infinite set S is countably infinite if there is a bijection from the set of Natural numbers onto the set S. \\
    $\Longleftarrow$ \\
    If set S has a bijection on the set of natural numbers $\mathbb{N}$ then by the definition of the cardinality, we have that $|\mathbb{N}| = |S|$. Thus S is countably infinite.
\end{proof}

\example{The sets $\mathbb{N}, \mathbb{Z}, \mathbb{Q}$ are countably infinite}
\[f: \mathbb{N} \to \mathbb{Z} \text{ via n } \to f(n) = \begin{cases}
    \dfrac{n}{2}, & \text{ n is even} \\
    -\dfrac{n+1}{2}, & \text{ n is odd}
\end{cases}\]
However, $\mathbb{R}^d, 2^d, [0, 1], [a, b]$ are countable.

\claim{$2^{\mathbb{N}}$ is not countable}
\begin{proof}
    Suppose we assume the contrary. Let $2^{\mathbb{N}}$ be countable set. \\
    Using the definition of the countable set, we know that there is a surjection of the set $2^{\mathbb{N}}$ to the set of Natural numbers. \\
    Then we have a mapping of a function f as, \[f: \mathbb{N} \to 2^{\mathbb{N}}\]
    We define $A := \{n \in \mathbb{N} | n \neq f(n)\}$ \\
    We must recall that f is one-to-one, that is there exists some m $\in \mathbb{N}$ such that f(m) = A. \\
    However, by the definition of A we have, $m \neq f(m)$.
    Here we have a contradiction. Thus, our assumption was incorrect. \\
    We conclude that $2^{\mathbb{N}}$ is not countable.
\end{proof}

\claim{[0, 1] is uncountable.}
\begin{proof}
    We will prove this by the method of contradiction. \\
    Suppose the set [0, 1] is countable. Then by definition, we can enumerate all the members of the set [0, 1] with natural numbers. \\
    So, \[[0, 1] = x_1, x_2, x_3, ..., \forall x \in [0, 1]\]
    [0, 1] has a decimal representation given as, 
    \[x = 0.b_1b_2b_3b_4.........\]
    Suppose there is an enumeration of all numbers, $x_1, x_2, x_3, .....$ in [0,1]
    \begin{align*}
        x_1 &= 0.b_{11}b_{12}b_{13}...................b_{1n}.... \\ 
        x_2 &= 0.b_{21}b_{22}b_{23}...................b_{2n}.... \\
        x_3 &= 0.b_{31}b_{32}b_{33}...................b_{3n}.... \\
        . \\
        . \\
        . \\
        x_n &= 0.b_{n1}b_{n2}b_{n3}...................b_{nn}.... 
    \end{align*}
    We can now construct a real number in [0, 1], which is not listed with the set of Natural numbers. \\
    Let $y =0.y_1y_2y_3...................y_n....$, such that we have, 
    \[y_1 \neq b_{11}\]
    \[y_2 \neq b_{22}\]
    \[y_3 \neq b_{33}\]
    \[y_4 \neq b_{44}\]
    \[.\]
    \[.\]
    \[.\]
    \[y_n \neq b_{nn}\]
    Here y is not equal to any of the number with two decimal representation, since $y_n \neq 0, 9.$
    Thus y and $x_n$ differ in the $n^{th}$ place, so $y_n \neq x_n$ for any $n \in \mathbb{N}$.
    Therefore, y is not included in the enumeration of [0,1] which is a contradiction to our statement.
    Hence, [0, 1] is uncountable.
\end{proof}

\definition{Let A, B be sets. Then. 
\begin{enumerate}
    \item A $\stackrel{def} {=} \text{ B iff }$ A $\subseteq$ B and B $\subseteq$ A, $\Longleftrightarrow$ ($\forall x \in A, \implies x \in B$ and $\forall x \in B \implies x \in A$) 
    \item $\varnothing:= \{ \}$ \textbf{empty} or \textbf{void} set which contains no element.
    \item A $\subsetneq$ B iff $A \subseteq$ B and $A \neq B.$
\end{enumerate}
}

\definition{Let A, B $\subseteq$ X be sets.} \\
The \textbf{Union} of A with B is given by, \[A \cup B := \{x \in X | x \in A \text{ or } x \in B\}\]
The \textbf{Intersection} of A and B is given by, \[A \cap B := \{x \in X | x \in A \text{ and } x \in B\}\]
The \textbf{Difference} of A and B is given by, \[A\backslash B:= \{x \in X | x \in A \text{ and } x \neq B\}\]
The \textbf{Symmetric-Difference} of A and B is defined as, \[A \triangle B:= (A \cup B)\backslash (A \cap B)\]
Two sets A and B are said to be \textbf{Disjoint} if and only if, \[A \cap B = \varnothing\]

\proposition{Distributive Laws}
\begin{align*}
    (i) (A \cup B) \cap C &= (A \cap C)\cup (B \cap C) \\
    (ii) (A \cap B) \cup C &= (A \cup C)\cap (B \cup C) \\
    (iii) (A \cup B) \backslash C &= (A \backslash C) \cup (B \backslash C) \\
    (iv) (A \cap B) \backslash C &= (A \backslash C) \cap (B \backslash C) 
\end{align*}
\begin{proof}
    (i) $(A \cup B) \cap C = (A \cap C)\cup (B \cap C)$ \\
    We will prove the above by the method of LHS contained in RHS and RHS contained in LHS. \\
    First we will prove that, $(A \cup B) \cap C \subseteq (A \cap C)\cup (B \cap C)$ (1)\\
    Let x $\in A \cap (B \cup C)$, then, 
    \begin{align*}
        x \in A \cap (B \cup C) &\Longrightarrow x \in A \text{ and } (x \in B \text{ or } x \in C)(\text{ Def. of Uni and Intersect} ) \\
        &\Longrightarrow (x \in A \text{ and } x \in B) \text{ or } (x \in A \text{ and } x \in C)(\text{Dist. law of logic}) \\
        &\Longrightarrow (x \in A \cap B) \text{ or } (x \in A \cap C) (\text{ Def of intersection}) \\
        &\Longrightarrow x \in (A \cap B) \cup (A \cap C) (\text{ Def of Unions.})
    \end{align*}
    Now we will prove the other way. 
    $(A \cap C)\cup (B \cap C) \subseteq (A \cup B) \cap C$ (2) \\
    Let x $\in (A \cap C)\cup (B \cap C)$, then, 
    \begin{align*}
        x \in (A \cap C)\cup (B \cap C) &\Longrightarrow (x \in A \cap B) \text{ or } (x \in A \cap C) (\text{Def. of Uni and Inter} ) \\ 
        &\Longrightarrow (x \in A \text{ and } x \in B) \text{ or } (x \in A \text{ and } x \in C) (\text{Dist. law}) \\
        &\Longrightarrow x \in A \text{ and } (x \in B \text{ or } x \in C) (\text{ Def of Union}) \\
        &\Longrightarrow x \in A \cap (B \cup C) (\text{ Def of Intersect.})
    \end{align*}
    From (1)and (2), we have that, $(A \cup B) \cap C = (A \cap C)\cup (B \cap C)$ \\
    Similarly one can show the problem (ii) as well.
\end{proof}
\begin{proof}
    (iii) We need to prove that $A \backslash (B \cup C) = (A \backslash B) \cap (A \backslash C)$ \\
    \begin{align*}
        A \backslash B \cup C &= A \backslash (B \cup C) \\
        &= A \cap (B \cup C)^{c} \\
        &= A \cap (B^{c} \cap C^{c}) \\
        &= (A \cap B^{c}) \cap (A \cap C^{c}) \\
        &= (A \backslash B) \cap (A \backslash C)
    \end{align*}
    Similarly one can prove (iv) as well.
\end{proof}

\definition{Let A $\subseteq$ X be a set. The \textbf{complement} of A, denoted as $A^{c}$ or ($\bar{A}$), is defined to be the set}\[A^c := X \backslash A; \hspace{4mm} ( =\{x \in X: x \notin A\})\]

\proposition{$\forall$ sets A, B $\subseteq$ X,}
\begin{align*}
    (A^{c})^{c} &= A, A \cap A^c = \varnothing, A \cup A^c = X, \\
    A \backslash B &= A \cap B^c, A \subseteq B \Longleftrightarrow B^c \subseteq A^c \\
    (A \cup B)^c &= A^c \cap B^c \\
    (A \cap B)^c &= A^c \cup B^c
\end{align*}
\begin{proof}
    (i) To prove that $(A^c)^c$, we need to verify the two containments, $(A^c)^c \subseteq A$ and $A \subseteq (A^c)^c$.\\
    We will begin by showing that $(A^c)^{c} \subseteq A.$ Suppose that $x \in (A^c)^c.$ By the definition of complement, this means that $x \neq (A^c).$ \\
    But this says precisely, that x is not in $A^c$, which by the definition of the complement again, means exactly that x is in A. In other words, x $\in A.$ \\
    To show that $A \subseteq (A^c)^c$. Assume that $x \in A$. \\
    By the definition of complement, this means that x is not in $(A^c).$ In other words, $x \in A^c$ so that $x \in (A^c)^c$.
\end{proof}
\begin{proof}
    (ii) Use the definition of Difference to show $A \backslash B = A \cap B^c$.
\end{proof}
\begin{proof}
    (iii) and (iv) The main idea of these proofs is that negation changes "and" into "or" and vice-versa. Therefore, we only prove the first law and second one follows the same. \\
    $\Longrightarrow$ Suppose x $\in (A \cap B)^c$. This means $x \notin (A \cap B)$. Notice that the negation of "$x \in A \text{ and } x \in B$" is equivalent to "$x \notin A \text{ or } x \notin B$". This implies that $x \in A^c \text{ or } x \in B^c$. In other words, $x \in A^c \cup B^c$. \\
    $\Longleftarrow$ Suppose x $\in A^c \cup B^c$. This means $x \notin A$ or $x \notin B.$ This is logically equivalent to the negation of "x $\in A$ and x $\in B$". In other words, it is equivalent to the negation of x $\in A \cap B$. We may conclude that $x \notin (A \cap B)$ that is, $x \in (A \cap B)^c.$ 
\end{proof}

\subsection{Indicator Functions, Set Limits, Convergence of Sets and Monotonicity}
\definition{Let A, X be sets with $A \subseteq$ X. The function $x \in X \to \mathds{I}_{A} : X \to \{0, 1\}$ defined by,}\[\mathds{I}_{A}(x) = \begin{cases}
    1 : x \in A \\
    0 : x \notin A
\end{cases}\]
is said to be the $\textbf{Indicator function}$ of A or $\textbf{Characteristic function}$ of A (relative to X).

\lemma{Let A, B $\subseteq$ X}, Then, 
\begin{enumerate}
    \item $A \subseteq B \Longleftrightarrow \mathds{I}_{A} \leq \mathds{I}_{B}$
    \begin{proof}
        For A, B $\in X$, $A \subseteq B \Longleftrightarrow \mathds{I}_{A} = \mathds{I}_{B}$ \\
        B can be written as, $B = (A \cap B) \cup (A^{c} \cap B)$ \\
        So, if x $\in A \cap B, \mathds{I}_{A} = \mathds{I}_{B} = 1$ \\
        If x $\in A^c \cap B, \mathds{I}_{A} = 0 < 1 = \mathds{I}_{B}$
        Hence, $\mathds{I}_{A} \leq \mathds{I}_{B}$ whenever, $A \subseteq B.$
    \end{proof}
    \item $\mathds{I}_{A \cap B} = \mathds{I}_{A} \cdot \mathds{I}_{B}$
    \begin{proof}
        For A, B $\in X$, we need to show that $\mathds{I}_{A \cap B} = \mathds{I}_{A} \cdot \mathds{I}_{B}$
        \begin{align*}
            \mathds{I}_{A \cap B} = 1 &\Longleftrightarrow x \in A \cap B \\
            &\Longleftrightarrow x \in A \text{ and } x \in B \\
            &\Longleftrightarrow \mathds{I}_{A} \cdot \mathds{I}_{B} = 1.
        \end{align*}
        Similarly, we can also show for the value 0.
        \begin{align*}
            \mathds{I}_{A \cap B} = 0 &\Longleftrightarrow x \in A \cap B \\
            &\Longleftrightarrow x \in A \text{ and } x \in B \\
            &\Longleftrightarrow \mathds{I}_{A} \cdot \mathds{I}_{B} = 0.
        \end{align*}
        Thus, $\mathds{I}_{A \cap B} = \mathds{I}_{A} \cdot \mathds{I}_{B}$
    \end{proof}
    \item $\mathds{I}_{A \cup B} = \mathds{I}_{A} + \mathds{I}_{B} - \mathds{I}_{A \cap B}$
    \begin{proof}
        For any A, B $\in X, \text{ We need to show that } \mathds{I}_{A \cup B} = \mathds{I}_{A} + \mathds{I}_{B} - \mathds{I}_{A}\cdot \mathds{I}_{B}$ \\
        Note that $A \cup B = (A \cup (A^c \cap B))$ and we can also write, B $= (A \cap B) \cup (A^c \cap B).$ 
        \[\mathds{I}_{B}(x) = \mathds{I}_{A \cap B} (x) + \mathds{I}_{A^c \cap B} (x)\]
        \[\mathds{I}_{A^c \cap B} (x) = \mathds{I}_{B}(x) - \mathds{I}_{A \cap B} (x)\]
        Now since $A \cup (A^c \cap B) = \varnothing,$ we have from the above equation,
        \begin{align*}
            \mathds{I}_{A \cup B} (x) &= \mathds{I}_{A \cup (A^c \cap B)}(x) = \mathds{I}_{A}(x) + \mathds{I}_{A^c \cap B}(x) \\
            &= \mathds{I}_{A}(x) + \mathds{I}_{B}(x) - \mathds{I}_{A \cap B}(x)
        \end{align*}
    \end{proof}
    \item $\mathds{I}_{A^c} = 1 - \mathds{I}_{A} = \mathds{I}_{X} - \mathds{I}_{A}$
    \begin{proof}
        For any A, $\in X,$ we know that $\mathds{I}_{A^c} = 1 - \mathds{I}_{A}$\\
        Then, if x $\in A^c, \mathds{I}_{A^c} = 1 = 1 - 0 = 1 - \mathds{I}_{A}$ (Since $\mathds{I}_A = 1$) \\
        If x $\notin A^c,$ which implies that $x \in A, \mathds{I}_{A^c} = 0 = 1 - 1 = 1 - \mathds{I}_{A}$
    \end{proof}
    \item $\mathds{I}_{A \backslash B} = \mathds{I}_{A} (1 - \mathds{I}_{B})$
    \begin{proof}
        The set difference is given by, $A \backslash B = A \cap B^c$
        \begin{align*}
            \mathds{I}_{A \backslash B} &= \mathds{I}_{A \cap B^c} \\
            &= \mathds{I}_{A} \cdot \mathds{I}_{B^c} \\
            &= \mathds{I}_{A} \cdot (1 - \mathds{I}_{B}) \\
            &= \mathds{I}_{A} (1 - \mathds{I}_{B}
        \end{align*}
    \end{proof}
    \item $\mathds{I}_{A \triangle B} = |\mathds{I}_{A} - \mathds{I}_{B}|$
    \begin{proof}
        The symmetric difference is given as, $A \triangle B = (A \backslash B) \cup (B \backslash A)$ 
        \begin{align*}
            \mathds{I}_{A \triangle B} &= \mathds{I}_{((A \backslash B) \cup (B \backslash A))} \\
            &=\mathds{I}_{A \backslash B} + \mathds{I}_{B \backslash A} \\
            &\leq |\mathds{I}_{A \backslash B}| + |\mathds{I}_{B \backslash A}| \\
            &=|\mathds{I}_{A \backslash B} - \mathds{I}_{B \backslash A}| \\
            &=|\mathds{I}_{A \backslash B} + \mathds{I}_{A \cap B} - \mathds{I}_{A \cap B} + \mathds{I}_{B \backslash A}| \\
            &=|\mathds{I}_{A} - \mathds{I}_{B}|
        \end{align*}
    \end{proof}
\end{enumerate}

\definition{Let ($A_n)_{n \in \mathbb{N}}$} be a sequence of sets $A_n \subseteq X$, Then, \[\underset{n \to +\infty}{\mathrm{lim sup}} A_n = \overset{\infty}{ \underset{n=1}{\cap}} \overset{\infty}{ \underset{k=n}{\cup}} A_{k} \textbf{ (Limes Superior}),\]
and 
\[\underset{n \to +\infty}{\mathrm{lim inf}} A_n = \overset{\infty}{ \underset{n=1}{\cup}} \overset{\infty}{ \underset{k=n}{\cap}} A_{k} \textbf{ (Limes Inferior}),\]
A sequence $(A_n)_{n \geq 1}$ is \textbf{convergent} $\Longleftrightarrow$ $\underset{n \to +\infty}{\mathrm{lim sup}} A_n = \underset{n \to +\infty}{\mathrm{lim inf}} A_n$ \\
$(A_n)_{n \geq 1}$ is \textbf{increasing} iff $\forall n \in \mathbb{N}, A_n \subseteq A_{n +1}.$ \\
Similarly, \textbf{Decreasing, Strictly Increasing, \text{ and } Strictly Decreasing} are defined as expected. \\
$(A_n)_{n \geq 1}$ is \textbf{Monotone} iff it is increasing or decreasing. \\

\remark{Trivially always, \[\underset{n \to +\infty}{\mathrm{lim inf}} A_n \subseteq \underset{n \to +\infty}{\mathrm{lim sup}} A_n\]}

\theorem{Every monotone sequence $(A_n)_{n \geq 1}$ with $A_n \subseteq X$ converges. Moreover, we have, \[{\underset{n \to + \infty}{lim}} A_{n} = \overset{\infty}{ \underset{n = 1}{\cup}} A_{n}\]}
if $(A_n)$ is increasing and, 
\[{\underset{n \to + \infty}{lim}} A_{n} = \overset{\infty}{ \underset{n = 1}{\cap}} A_{n}\]
if $(A_n)$ is decreasing.
\begin{proof}
    Suppose $(A_n)_{n \geq 1}$ is increasing. Then, \[\underset{n \to +\infty}{\mathrm{lim inf}} A_n = \overset{\infty}{ \underset{n=1}{\cup}} \overset{\infty}{ \underset{k=n}{\cap}} A_{k} = \overset{\infty}{ \underset{n = 1}{\cup}} A_{n} \supseteq \underset{n \to +\infty}{\mathrm{lim sup}} A_n \supseteq \underset{n \to +\infty}{\mathrm{lim inf}} A_n\]
    \[\Longrightarrow \underset{n \to +\infty}{\mathrm{lim inf}} A_n = \underset{n \to +\infty}{\mathrm{lim sup}} A_n\]
    Suppose $(A_n)_{n \geq 1}$ is decreasing. Then, \[\underset{n \to +\infty}{\mathrm{lim inf}} A_n = \overset{\infty}{ \underset{n=1}{\cup}} \overset{\infty}{ \underset{k=n}{\cap}} A_{k} = \overset{\infty}{ \underset{n=1}{\cup}} \overset{\infty}{ \underset{k=1}{\cap}} A_{k} = \overset{\infty}{ \underset{k=1}{\cap}} A_{k} = \overset{\infty}{ \underset{n=1}{\cap}} A_{n}  = \overset{\infty}{ \underset{n=1}{\cap}} \overset{\infty}{ \underset{k=n}{\cup}} A_{k} = \underset{n \to +\infty}{\mathrm{lim sup}} A_n\]
\end{proof}

\remark{Note that, \[\overset{\infty}{ \underset{n=1}{\cap}} \overset{\infty}{ \underset{k=n}{\cup}} A_{k} = \underset{n \to +\infty}{\mathrm{lim inf}} A_n = \{x \in X | x \in A_n, \text{ for infinitely many n } \in \mathbb{N}\},\]}
hence,
\[\underset{n \to +\infty}{\mathrm{lim sup}} A_n \{x \in X | \exists n_0 \in \mathbb{N}: \forall n \geq n_0(x), x \in A_n\}\]
For $(A_n)_{n \geq 1}$ increasing, \[A_n \uparrow A_{\infty}. \]
For $(A_n)_{n \geq 1}$ decreasing, \[A_n \downarrow A_{\infty}. \]
Similarly, 
$f_n \uparrow f_{\infty}$ and $f_n \downarrow f_{\infty}$ \\
For the real valued sequences, $(a_n)_{n \in \mathbb{N}}$, we have, 
\[\underset{n \to +\infty}{\mathrm{lim inf}} a_n = \underset{n \to +\infty}{\mathrm{lim inf}} \{a_k : k \geq n\} = \underset{k}{\mathrm{sup inf}} \{a_n : n \geq k\},\]
\[\underset{n \to +\infty}{\mathrm{lim sup}} a_n = \underset{n \to +\infty}{\mathrm{lim sup}} \{a_k : k \geq n\} = \underset{k}{\mathrm{inf sup}} \{a_n : n \geq k\}\]

\lemma{$\forall A_n, B_n, C \subseteq X$}
\begin{enumerate}
    \item $(\text{lim sup }_{n \to +\infty} A_n)^c = \text{ lim inf }_{n \to +\infty} A^{c}_{n}$
    \begin{proof}
        \begin{align*}
        (\text{lim sup }_{n \to +\infty} A_n)^c &= (\overset{\infty}{ \underset{n=1}{\cap}} \overset{\infty}{ \underset{k=n}{\cup}} A_k)^c \\
        &= \overset{\infty}{ \underset{n=1}{\cup}} (\overset{\infty}{ \underset{k=n}{\cup}} A_k)^c \\
        &= \overset{\infty}{ \underset{n=1}{\cup}} \overset{\infty}{ \underset{k=n}{\cap}} A_k^c \\
        &= {\underset{n \to + \infty}{\text{ lim inf }}} A_{k}^c 
        \end{align*}
    \end{proof}
    \item ${\mathds{I}_{\text{ lim inf }}}_{n \to + \infty} A_n = \text{ lim inf }_{n \to + \infty} \mathds{I}_{A_n}$ \newline
    ${\mathds{I}_{\text{ lim sup }}}_{n \to + \infty} A_n = \text{ lim sup }_{n \to + \infty} \mathds{I}_{A_n}$
    \begin{proof}
        
    \end{proof}
    \item $A_n \to C \Longleftrightarrow \mathds{I}_{A_n} \to \mathds{I}_{C}$
    \begin{proof}
        
    \end{proof}
    \item $\text{lim inf }_{n \to + \infty} A_n \cup  \text{lim sup }_{n \to + \infty} B_n \subseteq \text{lim sup }_{n \to + \infty} (A_n \cup B_n)$
    \begin{proof}
    
    \end{proof}
    \item $\text{lim sup }_{n \to + \infty} A_n \backslash \text{lim inf }_{n \to + \infty} A_n = \text{lim sup }_{n \to + \infty} (A_n \triangle A_{n+1}) $
    \begin{proof}
        
    \end{proof}
    \item $\exists \triangle^{\infty}_{n = 1} A_n = A_1 \triangle A_2 \triangle ... \Longleftrightarrow \text{lim}_{n \to + \infty} A_n = \varnothing$
    \begin{proof}
        
    \end{proof}
\end{enumerate}

\remark{($\mathcal{P}(X), \triangle$) forms an Abelian group (i.e. commutative and associative w.r.t $\triangle$ operations).}


\subsection{Family of Sets}
\definition{A family of sets is a non empty set F whose elements are sets by themselves, denoted by \[F = \{A_i | i \in I \} or \{A_i\}_{i \in I},\] where I is called the Index set or i Indices} 

\remark{Set operation are usually defined on F, for example, Unions \[\underset{i \in I} \cup A_i = \{x \in X | \exists i \in I \text{ such that } x \in X\}\]
and Intersections, \[\underset{i \in I} \cap A_i = \{x \in X | \forall i \in I \text{ such that } x \in X\}\]}

\theorem{Let $A_i$ and B be the subsets of X. Then,}
\begin{enumerate}
    \item $(\underset{i \in I} \cup A_i) \cap B = \underset{i \in I} \cup (A_i \cap B)$
    \begin{proof}
        Let x $\in \underset{i \in I} \cup A_i$ and $x \in B$, then, \\
        \[x \in (\underset{i \in I} \cup A_i) \cap B \Longleftrightarrow x \in A_i, \forall i \in I \text{ and } x \in B\]
        This implies that, \[x \in \underset{i \in I} \cup (A_i \cap B)\]
        Conversely, if \[x \in \underset{i \in I} (\cup) (A_i \cap B) \Longleftrightarrow x \in A_i \cap B, \forall i \in I\]
        This implies that \[x \in A_i \text{ and } x \in B\]
        then we have, \[x \in (\underset{i \in I} \cup A_i) \cap B\]
        Hence, we proved that \[(\underset{i \in I} \cup A_i) \cap B = \underset{i \in I} \cup (A_i \cap B)\]
    \end{proof}
    \item $(\underset{i \in I} \cap A_i) \cup B = \underset{i \in I} \cap (A_i \cup B)$ (Distributive Laws)
    \begin{proof}
        If x $\in B \cup (\underset{i \in I} \cap A_i),$ then $x \in B \text{ or } x \in \underset{i \in I} \cap A_i$. \\
        This implies that $x \in A_i \forall i \in I$ \\
        Thus, $x \in B \cup A_i, \forall i \in I$
    \end{proof}
    \item $(\underset{i \in I} \cup A_i)^c = \underset{i \in I} \cap A_i^c$ (De Morgan's Laws)
    \begin{proof}
        
    \end{proof}
    \item $(\underset{i \in I} \cap A_i)^c = \underset{i \in I} \cup A_i^c$ (De Morgan's Laws) 
    \begin{proof}
        
    \end{proof}
\end{enumerate}

\definition{Let $f: X \to Y$ be one-to-one and onto. Then $f^{-1}: Y \to X$ defined by \[f^{-1} (y) = x \Longleftrightarrow f(x) = y\] is called the inverse of f. Given two functions $f: X \to Y \text{ and } g: Y \to Z.$ Then $x \mapsto (g \circ f)(x) = g(f(x))$ is called the composition $g \circ f: X \to Z.$}

\theorem{On relationship between Images and Inverse Images. \\
Assume $(a_i)_{i \in I}$ is a family of subsets of X, $(B_i)_{i \in I}$ is a family of subsets of Y. Function $f: X \to Y,$ Then we have the following}























\subsection{Cartesian Product, Relation and Ordered Sets}

\subsection{Uncountable, Countable Sets, Cardinality as Subadditive Measure}

\subsection{Semiring, Ring, Algebra, $\sigma-$Algebra, and Borel Sets}

\subsection{Dynkin Systems and Minimal-Generated Systems}



\section[Chapter 2: Basics Concepts of Metric Spaces]{Metric Spaces}

\subsection{Basic concepts of Metric Spaces}

% --- Bibliography ---

% Start a bibliography with one item.
% Citation example: "\cite{williams}".

\begin{thebibliography}{1}

% \bibitem{williams}
%    Williams, David.
%    \textit{Probability with Martingales}.
%    Cambridge University Press, 1991.
%    Print.

% Uncomment the following lines to include a webpage
% \bibitem{webpage1}
%   LastName, FirstName. ``Webpage Title''.
%   WebsiteName, OrganizationName.
%   Online; accessed Month Date, Year.\\
%   \texttt{www.URLhere.com}

\end{thebibliography}

% --- Document ends here ---

\end{document}